Automatic garbage collection is widely used and has been shown to be very
efficient is given enough physical memory. With memory prices like to continue
their downwards trajectory of the last couple of decades, automatic garbage
collection is set to become even more competitive against manual memory management
in the future. However, memory bandwidth hasn't improved at the same pace as price
and latency has even gone up. This has lead to the development of memory caches
closer to the CPU but traditional garbage collection algorithms do not make good
use of these. This report explores two new techniques, named tail-copying and
tail-elimination, that extend Cheney's
semi-space garbage collection algorithm to improve cache friendliness and therefore
performance. Measurements of 7 benchmark programs shows that the two techniques
are able to reduce the number of cache misses but the performance benefits are
often lost due to increased branch misses.
